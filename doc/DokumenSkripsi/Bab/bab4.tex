\chapter{Perancangan}
\label{chap:perancangan}

Bab ini membahas tentang perancangan setiap fitur yang akan diimplementasi pada perangkat lunak \textit{Sharif Judge}. 

\section{Mengganti method \textit{shell\_exec("rm ...")} menjadi \textit{unlink()}}
\textit{Method} \textit{shell\_exec("rm ...")} yang memiliki fungsi untuk menghapus sebuah file terdapat pada file \textit{controller Assignment.php} tepatnya di baris kode 425 dan 473

\textit{Assignments.php}
\begin{lstlisting}[basicstyle=\ttfamily, frame=single,
columns=fullflexible, keepspaces=true]

...
423	// Upload Tests (zip file)
424	
425	shell_exec('rm -f '.$assignments_root.'/*.zip');
426	$config = array(
...
472	foreach($old_pdf_files as $old_name)
473		shell_exec("rm -f $old_name");
474	$this->messages[] = array(
...

\end{lstlisting}

Fungsi \textit{shell\_exec("rm ...")} pada baris 425 dan 473 akan diubah menggunakan fungsi \textit{unlink()} menjadi seperti berikut

\textit{Assignments.php}
\begin{lstlisting}[basicstyle=\ttfamily, frame=single,
columns=fullflexible, keepspaces=true]

...
423	// Upload Tests (zip file)
424	
425	unlink($assignments_root.'/*.zip');
426	$config = array(
...
472	foreach($old_pdf_files as $old_name)
473		unlink($old_name);
474	$this->messages[] = array(
...

\end{lstlisting}

\section{Menambahkan method rekoneksi ke \textit{database}}
\textit{Method} rekoneksi ke \textit{database} akan ditambahkan pada \textit{file controller Queueprocess.php}. 

\textit{Queueprocess.php}
\begin{lstlisting}[basicstyle=\ttfamily, frame=single,
columns=fullflexible, keepspaces=true]

...
133
134	// Save the result
135	$this->queue_model->save_judge_result_in_db($submission, $type);
...

\end{lstlisting}

\textit{Method} rekoneksi yang digunakan yaitu \textit{\$this->db->reconnect()}. \textit{Method} ini diletakan pada baris 134 tepat sebelum \textit{Sharif Judge} menyimpan hasil \textit{judge}. Hal tersebut dilakukan untuk menghindari \textit{connection times out} akibat pengujian yang memakan waktu lama.

\textit{Queueprocess.php}
\begin{lstlisting}[basicstyle=\ttfamily, frame=single,
columns=fullflexible, keepspaces=true]

...
133
134	//reconnect to database incase we have run test for a long time.
135	$this->db->reconnect();
136
137	// Save the result
138	$this->queue_model->save_judge_result_in_db($submission, $type);
...

\end{lstlisting}

\section{Membatasi soal (deskripsi \& PDF) hanya bisa diunduh saat \textit{assignment "open"} dan setelah waktu mulai}
Fungsi untuk mengunduh soal (deskripsi \& PDF) terdapat pada \textit{controller Assignment.php} tepatnya di baris kode 100.

\textit{Assignments.php}
\begin{lstlisting}[basicstyle=\ttfamily, frame=single,
columns=fullflexible, keepspaces=true, breaklines=true]

97	/**
98	* Download pdf file of an assignment (or problem) to browser
99	*/
100	public function pdf($assignment_id, $problem_id = NULL)
101	{
102		// Find pdf file
103		if ($problem_id === NULL)
104			$pattern = rtrim($this->settings_model->get_setting('assignments_root'),'/')."/assignment_{$assignment_id}/*.pdf";
105		else
106			$pattern = rtrim($this->settings_model->get_setting('assignments_root'),'/')."/assignment_{$assignment_id}/p{$problem_id}/*.pdf";
107		$pdf_files = glob($pattern);
108		if ( ! $pdf_files )
109			show_error("File not found");
110
111		// Download the file to browser
112		$this->load->helper('download')->helper('file');
113		$filename = shj_basename($pdf_files[0]);
114		force_download($filename, file_get_contents($pdf_files[0]), TRUE);
115	}

\end{lstlisting}

Selain membatasi soal (deskripsi \& PDF) hanya dapat diunduh saat \textit{assignment "open"} dan setelah waktu mulai, pada fungsi ini juga ditambahkan fitur lain. Fitur lain tersebut yaitu membatasi soal hanya dapat diunduh oleh peserta yang terdaftar sebagai "\textit{participant}" dan soal tidak dapat diunduh setelah melewati batas waktu pengumpulan. Rancangan algoritma kode yang akan digunakan yaitu
\begin{itemize}
	\item Membuat atribut tambahan untuk menyimpan informasi waktu selesai, waktu mulai dan waktu tambahan sebuah assignment.
	\item Jika atribut "\textit{open}" pada \textit{assignment} tidak memiliki nilai, maka munculkan pesan \textit{error} "\textit{Selected assignment has been closed}."
	\item Jika pengguna tidak terdaftar sebagai "\textit{participant}" dalam \textit{assignment} yang dipilih, maka munculkan pesan error "\textit{You are not registered for submitting}."
	\item Jika waktu sekarang telah melewati batas waktu selesai + waktu tambahan, maka munculkan pesan \textit{error} "\textit{Selected assignment has finished}."
	\item Jika waktu sekarang belum melewati waktu mulai, maka munculkan pesan \textit{error} "\textit{Selected assignment has not started}."	
\end{itemize}

Berikut hasil pengimplementasian rancangan algoritma di atas ke dalam kode program

\textit{Assignments.php}
\begin{lstlisting}[basicstyle=\ttfamily, frame=single,
columns=fullflexible, keepspaces=true, breaklines=true]

...
97	/**
98	* Download pdf file of an assignment (or problem) to browser
99	*/
100	public function pdf($assignment_id, $problem_id = NULL)
101	{
102		$finishtime = strtotime($this->assignment_model->assignment_info($assignment_id)['finish_time']);
103		$starttime = strtotime($this->assignment_model->assignment_info($assignment_id)['start_time']);
104		$extratime = $this->assignment_model->assignment_info($assignment_id)['extra_time'];
105
106		// Find pdf file
107		if ($problem_id === NULL)
108			$pattern = rtrim($this->settings_model->get_setting('assignments_root'),'/')."/assignment_{$assignment_id}/*.pdf";
109		else
110			$pattern = rtrim($this->settings_model->get_setting('assignments_root'),'/')."/assignment_{$assignment_id}/p{$problem_id}/*.pdf";
111		$pdf_files = glob($pattern);
112		if ( ! $pdf_files )
113			show_error("File not found");
114		elseif (!$this->assignment_model->assignment_info($assignment_id)['open'])
115			show_error('Selected assignment has been closed.');
116		elseif	( ! $this->assignment_model->is_participant($this->assignment_model->assignment_info($assignment_id)['participants'],$this->user->username) )
117			show_error('You are not registered for submitting.');
118		elseif ( shj_now() > $finishtime + $extratime)
119			show_error('Selected assignment has finished.');
120		elseif ( shj_now() < $starttime)
121			show_error('Selected assignment has not started.');
122			
123		// Download the file to browser
124		$this->load->helper('download')->helper('file');
125		$filename = shj_basename($pdf_files[0]);
126		force_download($filename, file_get_contents($pdf_files[0]), TRUE);
127	}
...

\end{lstlisting}

\section{Mensupport \textit{file} dengan ekstensi TXT}
Untuk dapat mensupport \textit{file} dengan ekstensi TXT pada perangkat lunak \textit{Sharif Judge}, diperlukan penambahan dan perubahan kode pada beberapa \textit{file}. Beberapa \textit{file} tersebut antara lain \textit{controller Submit.php, model Assignment\_model.php, view submissions.twig} dan \textit{file} bantuan \textit{shj\_helper.php} yang terdapat pada direktori \path{Sharif-Judge\helpers\}. Berikut beberapa baris potongan kode program

\textit{Submit.php}
\begin{lstlisting}[basicstyle=\ttfamily, frame=single,
columns=fullflexible, keepspaces=true, breaklines=true]

...
58		case 'java': return 'java';
59		case 'zip': return 'zip';
60		case 'pdf': return 'pdf';
61		default: return FALSE;
62	}
...
76		case 'java': return ($extension==='java'?TRUE:FALSE);
77		case 'zip': return ($extension==='zip'?TRUE:FALSE);
78		case 'pdf': return ($extension==='pdf'?TRUE:FALSE);
79	}
...
88	if ($str=='0')
89		return FALSE;
90	if (in_array( strtolower($str),array('c', 'c++', 'python 2', 'python 3', 'java', 'zip', 'pdf')))
91		return TRUE;
92	return FALSE;
...

\end{lstlisting}

\textit{Assignment\_model.php}
\begin{lstlisting}[basicstyle=\ttfamily, frame=single,
columns=fullflexible, keepspaces=true, breaklines=true]

...
100	$item2 = strtolower($item);
101	if ( ! in_array($item2, array('c','c++','python 2','python 3','java','zip','pdf')))
102		continue;
...

\end{lstlisting}

\textit{shj\_helper.php}
\begin{lstlisting}[basicstyle=\ttfamily, frame=single,
columns=fullflexible, keepspaces=true, breaklines=true]

...
81	case 'java': return 'java';
82	case 'zip': return 'zip';
83	case 'pdf': return 'pdf';
84	default: return FALSE;
...
104	case 'java': return 'Java';
105	case 'zip': return 'Zip';
106	case 'pdf': return 'PDF';
107	default: return FALSE;
...

\end{lstlisting}

\textit{submissions.twig}
\begin{lstlisting}[basicstyle=\ttfamily, frame=single,
columns=fullflexible, keepspaces=true, breaklines=true]

...
160	<td>
161		
162			<div class="btn shj-orange" data-type="download">Download</div>
163		
164			<div class="btn shj-orange" data-type="code" >Code</div>
165		
166	</td>
...

\end{lstlisting}

Penambahan dan perubahan kode dilakukan setelah baris 60, 78 dan 90 pada \textit{controller Submit.php}. Berikut hasil penambahan dan perubahan kode 

\textit{Submit.php}
\begin{lstlisting}[basicstyle=\ttfamily, frame=single,
columns=fullflexible, keepspaces=true, breaklines=true]

...
58		case 'java': return 'java';
59		case 'zip': return 'zip';
60		case 'pdf': return 'pdf';
61		case 'txt': return 'txt';
62		default: return FALSE;
63	}
...
77		case 'java': return ($extension==='java'?TRUE:FALSE);
78		case 'zip': return ($extension==='zip'?TRUE:FALSE);
79		case 'pdf': return ($extension==='pdf'?TRUE:FALSE);
80		case 'txt': return ($extension==='txt'?TRUE:FALSE);
81	}
...
90	if ($str=='0')
91		return FALSE;
92	if (in_array( strtolower($str),array('c', 'c++', 'python 2', 'python 3', 'java', 'zip', 'pdf', 'txt')))
93		return TRUE;
94	return FALSE;
...

\end{lstlisting}

Perubahan kode dilakukan di baris 101 pada model \textit{Assignment\_model.php}. Berikut hasil perubahan kode 

\textit{Assignment\_model.php}
\begin{lstlisting}[basicstyle=\ttfamily, frame=single,
columns=fullflexible, keepspaces=true, breaklines=true]

...
100	$item2 = strtolower($item);
101	if ( ! in_array($item2, array('c','c++','python 2','python 3','java','zip','pdf','txt')))
102		continue;
...

\end{lstlisting}

Penambahan kode dilakukan setelah baris 83 pada \textit{file} bantuan \textit{shj\_helper.php}. Berikut hasil penambahan kode 

\textit{shj\_helper.php}
\begin{lstlisting}[basicstyle=\ttfamily, frame=single,
columns=fullflexible, keepspaces=true, breaklines=true]

...
81	case 'java': return 'java';
82	case 'zip': return 'zip';
83	case 'pdf': return 'pdf';
84	case 'txt': return 'txt';
85	default: return FALSE;
...
105	case 'java': return 'Java';
106	case 'zip': return 'Zip';
107	case 'pdf': return 'PDF';
108	case 'txt': return 'TXT';
109	default: return FALSE;
...

\end{lstlisting}

Perubahan kode dilakukan pada baris 161 pada \textit{view submissions.twig}. Berikut hasil perubahan kode

\textit{submissions.twig}
\begin{lstlisting}[basicstyle=\ttfamily, frame=single,
columns=fullflexible, keepspaces=true, breaklines=true]

...
160	<td>
161		
162			<div class="btn shj-orange" data-type="download">Download</div>
163		
164			<div class="btn shj-orange" data-type="code" >Code</div>
165		
166	</td>
...

\end{lstlisting}

\section{Membuat halaman \textit{Logs} yang mencatat aktivitas \textit{login} pengguna}
Agar halaman \textit{Logs} dapat berjalan dengan baik, perlu ditambahkan tabel baru pada \textit{database} \textit{Sharif Judge}.  \textit{Tabel} baru tersebut akan bernama \textit{shj\_logins}. 
\begin{table}[H] %atau h saja untuk "kira kira di sini"
	\centering 
	\caption{Perancangan Tabel \textit{shj\_logins}}
	\label{tab:tabellogs}
		\begin{tabular}{|c|c|c|c|}
			\hline
			\textbf{Atribut} & \textbf{Tipe Data} & \textbf{Ukuran}  & \textbf{Default} \\
			\hline
			\textit{login\_id} (PK*) & int & 11  & None \\
			\hline
			\textit{username} & varchar & 20  & None \\
			\hline
			\textit{ip\_address} & varchar & 15  & None \\
			\hline
			\textit{timestamp} & timestamp & 11  & current\_timestamp \\
			\hline
			\textit{last\_24h\_login\_id}	 & int & 11  & null \\
			\hline
		\end{tabular}
\end{table}
*PK = \textit{Primary Key}.

Keterangan atribut:
\begin{enumerate}
	\item \textit{login\_id}: sebagai penanda yang membedakan setiap \textit{login} peserta satu dengan yang lain. Memiliki \textit{length default} int dari \textit{phpMyAdmin} yaitu 11. Atribut \textit{login\_id} merupakan \textit{primary key} karena id harus unik agar setiap \textit{login} peserta dapat dibedakan. Atribut ini juga bersifat \textit{auto increment}.
	\item \textit{username}: \textit{username} peserta yang berhasil \textit{login} pada \textit{Sharif Judge}. Memiliki \textit{length varchar} 20 karena \textit{length username} pada tabel \textit{shj\_users} adalah 20.
	\item \textit{ip\_address}: \textit{ip address} peserta yang berhasil \textit{login} pada \textit{Sharif Judge}. Memiliki \textit{length varchar} 15 karena \textit{length} maksimal dari \textit{ip address protocol version 4 (IPv4)} adalah 15. Contoh: 202.100.123.255
	\item \textit{timestamp}: waktu peserta saat berhasil \textit{login} pada \textit{Sharif Judge}. Menggunakan tipe data timestamp yang akan mencatat waktu \textit{login} dengan format YYYY-MM-DD HH:MM:SS. Contoh: 2018-04-06 18:15:43
	\item \textit{last\_24h\_login\_id}: id \textit{login} peserta yang berhasil \textit{login} pada \textit{Sharif Judge} namun menggunakan \textit{ip address} berbeda dalam waktu 24 jam terakhir.
\end{enumerate}

Selain tabel diatas, halaman logs juga akan ditambahkan \textit{model, view} dan \textit{controller}.
\begin{enumerate}
	\item \textit{Model} \\
	\textit{Model} untuk halaman \textit{logs} akan bernama \textit{Logs\_model.php}. Berikut adalah perincian fungsi yang terdapat dalam rancangan \textit{model Logs\_model.php}.
	\begin{table}[H]
		\caption{Perincian fungsi \textit{insert\_to\_logs}}
		\begin{tabular}{|c|p{11cm}|}
			\hline
			Nama \textit{Method} 	& 	\textit{insert\_to\_logs} 	\\
			\hline
			Parameter \textit{Input} & \textit{\$username} dan \textit{\$ip\_address} \\
			\hline
			Parameter \textit{Output} & -\\
			\hline
			Tabel yang berhubungan & \textit{shj\_logins} \\
			\hline
			Deskripsi	& Proses untuk memasukan \textit{logs} pengguna \textit{Sharif Judge} \\
			\hline
			Algoritma	& \begin{itemize}
				\item mengecek dan menghapus \textit{logs} pada tabel \textit{shj\_logins} yang \textit{timestampnya} lebih dari 24 jam
				\item mengecek entri \textit{login} terakhir untuk \textit{\$username} yang  menggunakan \textit{IP address} tidak sama dengan \textit{\$ip\_address}
				\item jika tidak memiliki hasil, maka tambahkan entri baru menggunakan \textit{\$username} dan \textit{\$ip\_address} tersebut
				\item jika memiliki hasil,  maka tambahkan entri baru menggunakan \textit{\$username} dan \textit{\$ip\_address} serta last\textit{\_24h\_login\_id} diisi dengan \textit{login\_id} sebelumnya
			\end{itemize} \\
			\hline
		\end{tabular}
	\end{table}

	\begin{table}[H]
		\caption{Perincian fungsi \textit{get\_all\_logs}}
		\begin{tabular}{|c|p{11cm}|}
			\hline
			Nama \textit{Method} 	& 	\textit{get\_all\_logs} 	\\
			\hline
			Parameter \textit{Input} & - \\
			\hline
			Parameter \textit{Output} &  semua entri logs dari tabel \textit{shj\_logins}\\
			\hline
			Tabel yang berhubungan & \textit{shj\_logins} \\
			\hline
			Deskripsi	& Proses untuk mengembalikan entri \textit{logs} yang terdapat pada tabel \textit{shj\_logins} \\
			\hline
			Algoritma	& \begin{itemize}
				\item mengembalikan seluruh entri logs yang terdapat pada tabel \textit{shj\_logins} dalam bentuk \textit{array}
			\end{itemize} \\
			\hline
		\end{tabular}
	\end{table}

	\item View \\
	\textit{View} untuk halaman \textit{logs} akan bernama \textit{Logs.twig}. Menu halaman \textit{logs} akan terletak di bawah menu \textit{Scoreboard} dan akan bernama '\textit{24-hour log}'. Berikut adalah rancangan tampilan halaman \textit{logs}.
	
	\begin{figure}[H]
		\centering  
		\includegraphics[width=1.0\textwidth]{mockuplogs}  
		\caption[Rancangan tampilan halaman \textit{logs}]{Rancangan tampilan halaman \textit{logs}} 
		\label{fig:mockuplogs} 
	\end{figure}

	\item \textit{Controller} \\
	\textit{Controller} untuk halaman \textit{logs} akan bernama \textit{Logs.php}. Berikut adalah perincian fungsi yang terdapat dalam rancangan \textit{controller Logs.php}.
	\begin{table}[H]
		\caption{Perincian fungsi \textit{consturct\_\_}}
		\begin{tabular}{|c|p{11cm}|}
			\hline
			Nama \textit{Method} 	& 	\textit{consturct\_\_} 	\\
			\hline
			Parameter \textit{Input} & - \\
			\hline
			Parameter \textit{Output} &  - \\
			\hline
			Tabel yang berhubungan & - \\
			\hline
			Deskripsi	& membatasi pengguna yang dapat mengakses halaman \textit{logs}	 \\
			\hline
			Algoritma	& \begin{itemize}
				\item mengecek \textit{session} pengguna yang akan mengakses halaman \textit{logs}
				\item jika \textit{session} tidak berstatus '\textit{logged\_in}, maka pengguna akan dialihkan ke halaman \textit{login}
				\item mengecek \textit{role} pengguna yang akan mengakses halaman \textit{logs}
				\item jika role pengguna bukan \textit{admin}, maka pengguna akan dialihkan ke halaman \textit{'404 Not Found'}
			\end{itemize} \\
			\hline
		\end{tabular}
	\end{table}
	
	\begin{table}[H]
		\caption{Perincian fungsi \textit{index}}
		\begin{tabular}{|c|p{11cm}|}
			\hline
			Nama \textit{Method} 	& 	\textit{index} 	\\
			\hline
			Parameter \textit{Input} & - \\
			\hline
			Parameter \textit{Output} &  - \\
			\hline
			Tabel yang berhubungan & \textit{shj\_logins} \\
			\hline
			Deskripsi	& Proses untuk memuat seluruh entri \textit{logs} pada halaman \textit{Logs.twig}	 \\
			\hline
			Algoritma	& \begin{itemize}
				\item memuat data \textit{logs} menggunakan fungsi \textit{get\_all\_logs} dari \textit{model Logs\_model.php}
				\item memproses data untuk tampilan \textit{Logs.twig}
			\end{itemize} \\
			\hline
		\end{tabular}
	\end{table}
\end{enumerate}
