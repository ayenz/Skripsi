\chapter{Implementasi dan Pengujian}
\label{chap:implementasi dan pengujian}

Bab ini membahas tentang implementasi dan pengujian perangkat lunak berdasarkan rancangan yang sudah dibuat. Ada dua jenis pengujian yang dilakukan, yaitu pengujian fungsional dan pengujian eksperimental. Bab ini juga membahas tentang lingkungan yang digunakan untuk pengujian perangkat lunak ini.

\section{Lingkungan untuk Pengujian}
Pengujian fungsional dan pengujian eksperimental dilakukan menggunakan dua jenis lingkungan yang berbeda. 
\begin{enumerate}
	\item Pengujian Fungsional. \\
	Berikut spesifikasi perangkat keras dan perangkat lunak yang digunakan untuk melakukan pengujian fungsional
	\begin{table}[H] %atau h saja untuk "kira kira di sini" 
		\caption{Lingkungan perangkat keras untuk pengujian fungsional}
		\label{tab:lingkunganpkpf}
		\resizebox{\textwidth}{!}{%
			\begin{tabular}{|l|l|}
				\hline
				Parameter & Nilai \\
				\hline
				\hline
				Processor & Intel Core i5 4200u\\
				\hline
				Graphics Processing Unit (GPU) & Intel HD Graphics HD4000 dan Nvidia GeForce 840M\\
				\hline
				Random Access Memory (RAM)& 12.00GB DDR3\\
				\hline
				Storage & 120GB SSD dan 1TB Harddisk\\
				\hline
		\end{tabular}}
	\end{table}

	\begin{table}[H] %atau h saja untuk "kira kira di sini" 
		\caption{Lingkungan perangkat lunak untuk pengujian fungsional}
		\label{tab:lingkunganplpf}
		\resizebox{\textwidth}{!}{%
			\begin{tabular}{|l|l|}
				\hline
				Parameter & Nilai \\
				\hline
				\hline
				Sistem Operasi & Windows 10 Education 64-bit\\
				\hline
				Bahasa Pemrograman & PHP, JavaScript, CSS dan HTML\\
				\hline
				Text Editor & Atom\\
				\hline
				Framework & CodeIgniter dan Twig\\
				\hline
				\multirow{4}{*}{Perangkat Lunak lainnya} 	& XAMPP Control Panel v3.2.2\\
															& Google Chrome Version 65.0.3325.181 (Official Build) (64-bit)\\
															& Firefox Quantum 59.0.2 (64-bit)\\
															& Microsoft Excel 2016\\
				\hline
		\end{tabular}}
	\end{table}
\end{enumerate}