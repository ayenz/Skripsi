%versi 2 (8-10-2016) 
\chapter{Pendahuluan}
\label{chap:intro}
   
\section{Latar Belakang}
\label{sec:label}
\textit{Online judge} merupakan sebuah sistem \textit{online} yang dirancang untuk mengevaluasi kode algoritma yang dikumpulkan oleh pengguna. Sistem dapat mengcompile dan mengeksekusi kode serta mengujinya ke dalam lingkungan homogen menggunakan data yang telah disediakan. Kode yang dikumpulkan dapat dijalankan dengan batasan-batasan seperti \textit{time limit, memory limit,} keamanan dan sebagainya. Penggunaan \textit{online judge} sering ditemukan pada kompetisi pemrograman dan mata kuliah pemrograman \cite{wasik:18:survey}.

\textit{Sharif Judge} adalah \textit{online judge} gratis untuk bahasa pemrograman C, C++, Java dan Python. Perangkat lunak ini diciptakan oleh Mohammad Javad Naderi pada tahun 2014 dan bersifat \textit{open source}. Antarmuka \textit{Sharif Judge} ditulis menggunakan bahasa pemrograman PHP (\textit{framework CodeIgniter}) dan \textit{backend} menggunakan \textit{BASH} \cite{mjnaderi:14:sharifjudge}.

Fungsi utama \textit{Sharif Judge} adalah menilai jawaban yang telah dikumpulkan oleh para peserta secara otomatis. Cara kerja penilaian otomatis pada \textit{Sharif Judge} dimulai dari dosen membuat \textit{assignment}. Untuk membuat \textit{assignment} dibutuhkan data-data seperti nama \textit{assignment}, waktu mulai, waktu berhenti, waktu tambahan, daftar peserta, deskripsi soal dan kunci jawaban. Para peserta dapat mengunduh deskripsi soal lalu mengerjakan \textit{assignment} tersebut. Peserta yang telah selesai mengerjakan \textit{assignment} dapat mengumpulkan jawaban dalam bentuk kode program. \textit{Sharif Judge} akan menjalankan kode program, membandingkan dengan kunci jawaban. Setelah menyesuaikan hasil keluaran dari kode program dengan kunci jawaban, \textit{Sharif Judge} akan menilai jawaban peserta.

\textit{Sharif Judge} digunakan oleh Program Studi Teknik Informatika Universitas Katolik Parahyangan pada mata kuliah seperti Algoritma dan Struktur Data serta Desain dan Analisis Algoritma. Perangkat lunak \textit{Sharif Judge} sangat membantu dosen dan mahasiswa dalam bidang akademik. Sistem penilaian otomatis merupakan salah satu fitur yang sering digunakan oleh para dosen. Dengan memanfaatkan fitur di atas, dosen dapat dengan mudah memberikan nilai tugas, kuis dan ujian. Mahasiswa juga dapat melihat nilai secara langsung setelah jawaban dikumpulkan. Para mahasiswa dapat memperbaiki jawaban yang telah dikumpulkan jika \textit{assignment} yang dikerjakan tidak melewati batas waktu pengumpulan.

Pada prakteknya, perangkat lunak \textit{Sharif Judge} versi terkini masih butuh pengembangan. Pengembangan tersebut dibutuhkan karena Program Studi Teknik Informatika memiliki kebutuhan yang lebih spesifik. Kebutuhan spesifik tersebut seperti \textit{login} yang terintegrasi dengan server RADIUS Teknik Informatika, membatasi pengaksesan deskripsi soal pada assignment dan kebutuhan spesifik lainnya. \textit{Sharif Judge} terakhir dicommit di \textit{GitHub} pada bulan Juli 2015 dan masih memiliki beberapa \textit{bug} yang belum diperbaiki. Hal-hal di atas menyebabkan \textit{Sharif Judge} kurang memenuhi kebutuhan Program Studi Teknik Informatika. 

Pada skripsi ini, akan dikemnbangkan \textit{Sharif Judge} agar sesuai dengan kebutuhan yang disebutkan di atas. Dari kebutuhan yang disebutkan di atas, akan dirancang fitur-fitur untuk diimplementasikan pada \textit{Sharif Judge}. Dengan pengimplementasian fitur yang baru, diharapkan kebutuhan mahasiswa dan dosen dapat terpenuhi.

\section{Rumusan Masalah}
\label{sec:rumusan}
\begin{enumerate}
	\item Fitur-fitur apa saja yang dibutuhkan oleh Teknik Informatika?
	\item Bagaimana mengembangkan Sharif Judge sehingga memenuhi kebutuhan Teknik Informatika?
\end{enumerate}

\section{Tujuan}
\label{sec:tujuan}
\begin{enumerate}
	\item Menganalisa dan mengetahui fitur-fitur yang dibutuhkan Teknik Informatika.
	\item Mengimplementasi kebutuhan Program Studi Teknik Informatika pada \textit{Sharif Judge}.
\end{enumerate}

\section{Batasan Masalah}
\label{sec:batasan}
Batasan masalah yang dibuat terkait dengan pengerjaan skripsi ini adalah sebagai berikut :
\begin{enumerate}
	\item Perangkat lunak diuji pada mata kuliah AIF102 (Algoritma \& Sturktur Data).
\end{enumerate}

\section{Metodologi}
\label{sec:metlit}Metodologi yang dilakukan dalam pengerjaan skripsi ini adalah sebagai berikut :

\begin{enumerate}
	\item Studi literatur mengenai:
		\begin{itemize}
			\item \textit{CodeIgniter} sebagai \textit{framework} untuk mengembangkan perangkat lunak.
			\item Dokumentasi \textit{Sharif Judge} sebagai panduan untuk mengembangkan perangkat lunak.
		\end{itemize}
	\item Menganalisis kebutuhan-kebutuhan dari para dosen pengguna \textit{Sharif Judge} dan daftar \textit{issue} pada repositori \textit{Sharif Judge} pada \textit{Github}.
	\item Menentukan dan merancang fitur yang akan diimplementasikan.
	\item Mengimplementasikan fitur terhadap perangkat lunak.
	\item Menguji perangkat lunak ke mata kuliah selama satu semester.
	\item Membuat dokumentasi perangkat lunak.
\end{enumerate}

\section{Sistematika Pembahasan}
\label{sec:sispem}
Setiap bab dalam skripsi ini memiliki sistematika penulisan yang dijelaskan ke dalam poin-poin sebagai berikut:

\begin{enumerate}
	\item Bab 1 : Pendahuluan \\
	Bab 1 membahas mengenai gambaran umum penelitian ini. Berisi tentang latar belakang, rumusan masalah, tujuan, batasan masalah, metodologi dan sistematika pembahasan.
	
	\item Bab 2 : Landasan Teori \\
	Bab 2 membahas mengenai teori-teori yang mendukung berjalannya penelitian ini serta tentang \textit{CodeIgniter} dan dokumentasi \textit{Sharif Judge}.
	
	\item Bab 3 : Analisis \\
	Bab 3 membahas mengenai analisis fitur-fitur yang akan diimplementasi pada \textit{Sharif Judge}.
	
	\item Bab 4 : Perancangan \\
	Bab 4 membahas mengenai perancangan yang dilakukan sebelum masuk ke tahap implementasi.
	
	\item Bab 5 : Implementasi dan Pengujian \\
	Bab 5 membahas mengenai implementasi dan pengujian yang telah dilakukan.
	
	\item Bab 6 : Kesimpulan dan Saran \\
	Bab 6 membahas hasil kesimpulan dari keseluruhan penelitian ini dan saran-saran yang dapat diberikan untuk penelitian berikutnya.
\end{enumerate}