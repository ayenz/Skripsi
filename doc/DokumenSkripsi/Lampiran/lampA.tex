%versi 3 (18-12-2016)
\chapter{Kode Program Halaman \textit{24-hour Logs}}
\label{lamp:kodeprogramhalamanlogs}

%terdapat 2 cara untuk memasukkan kode program
% 1. menggunakan perintah \lstinputlisting (kode program ditempatkan di folder yang sama dengan file ini)
% 2. menggunakan environment lstlisting (kode program dituliskan di dalam file ini)
% Perhatikan contoh yang diberikan!!
%
% untuk keduanya, ada parameter yang harus diisi:
% - language: bahasa dari kode program (pilihan: Java, C, C++, PHP, Matlab, C#, HTML, R, Python, SQL, dll)
% - caption: nama file dari kode program yang akan ditampilkan di dokumen akhir
%
% Perhatian: Abaikan warning tentang textasteriskcentered!!
%

Kode Program kelas \textit{model} halaman \textit{24-hour Logs}.
\lstinputlisting[language=php, caption=\textit{Logs\_model.php}]{./Lampiran/Logs_model.php} 

Kode Program kelas \textit{view} halaman \textit{24-hour Logs}.
\lstinputlisting[language=html, caption=\textit{logs.twig}]{./Lampiran/logs.twig} 

Kode Program kelas \textit{controller} halaman \textit{24-hour Logs}.
\lstinputlisting[language=php, caption=\textit{Logs.php}]{./Lampiran/Logs.php}
 

